\documentclass{article}

\usepackage{amsmath}
\usepackage{appendix}
\usepackage{amsthm}

\theoremstyle{plain}
\newtheorem{theo}{Theorem}[section]

\begin{document}

\title{Madlang - a DSL for Markov Chains}
\author{Vanessa McHale}
\maketitle

\begin{abstract}
We present madlang, a DSL for easily readable markov chains and templating
random text.
\end{abstract}

\tableofcontents

\section{Introduction}

\subsection{Definitions}

\newtheorem{defin}[theo]{Definition} goes here??

\subsection{Generating random text with Markov Chains}

% example from some vaguely famous book? 

\section{Markov Chains to generate text}

\subsection{Human-readable markov chains}

\subsection{Twitter bots}

\section{Templating Text}

\begin{appendices}
    \section{Language specifications}

    \section{Interpreter mechanics}
    
    \subsection{Parser}

    \subsection{Evaluation}
\end{appendices}

% example out of magical realism bot? or the like. 

\end{document}
